\documentclass[12pt,a4paper]{report}

% to reduce indent
\setlength{\parindent}{0pt}
%\setlength{\parskip}{1ex plus 0.5ex minus 0.2ex}
%

\title{Product Documentation}

\begin{document}
\section*{Foreword}
This document is part of a bachelor project developed at Oslo Univerisity College, department for engineering, spring 2008.

This product document for `Vizualizing Configuration System Data' will describe the final product, and it's uses and structure. Combined with the process documentation it will give a helhetlig impression of the work which has been done and the results of the product which has been accomplished.

The product document is mainly written for the project assigner, our local school mentor, and the sensor evaluating the project. Even though this document explains the product in a simple way, some basic knowledge about the field is required to fully understand the contents of this document.

%TODO: Write about what exactly is in this document : e.g. what technologies have been used, system structure etc

\section*{Description of the program}

LCFGvisualizer is a program..

The program is divided into two separate parts:
dataImporter and dataVisualizer (figure ..) image

1. dataImporter
This is the part used to parse the XML-profiles, extract the wanted information and export it to a mySQL database. 

2. dataVisualizer
This part connects to the database and generates a visualisation based on which criterias and techiques the user sets.
\tableofcontents
\chapter{User guide}
User guide will be added here...

\section{Installing lcfgVisualizer}
installation guide.. 
\section{Importing XML data}
\paragraph{Prerequisites:}
Before extracting data into the database, some things should be prepared in a proper way:
The \textbf{name} of the XML file in conjunction with the \textbf{lcfg/last\_modified} field will be used as primary keys in the database.

The XML-files you want to import should each lie in a separate directory for each of the dates.

A database server should be up and running, preferrably a mySQL database.
Create an empty table to hold imported data.
Make sure the system requirements are met (see SRS section further down)

Edit the config file cfg/vcsd.cfg to fit your system and needs.
In specific, make sure the database section matches your database server and tablename.
For further info about vcsd.cfg, see section \ref{configfile} .

\subsection{Initial import}
When the config file is fully configured, and all the desired components have been declared, the extracting of data into the database can begin. Run the perl file XML\_to\_DB.pl in a terminal or on a commmand line. If everything is configured correctly, this should go smoothly. For each main component (e.g. $<$inv$>$ ) this operation takes approximately one minute on a system with 1000 XML files, depending on the Hardware and LAN/Internet speed if the database is externally located.

The script will print out the parameters and child values chosen, confirm by pressing enter for each main component.
The script will print out the number of files found in the folder, push any key to continue the import.
If some XML-files is not well-formed or have errors, the script will print a warning and the file will not be imported.
When done, the script will print out the number of errors encountered ( if any ) and the time elapsed.
You have now imported a dataset, and may go on to adding further data sets \ref{append}, or skipping to the visualisation part \ref{visualisation}.

\subsection{Appending newer data sets}
\label{append}
To append more datasets to the database, just edit vcsd.cfg and set the xmlpath-variable to reflect the folder you want to import.
It is wise (but not necessary) to keep all the components that were used in an initial import.
If you add further components that were not imported initally, the initial values will be set to 'unknown'. %skrive litt smartere om dette.. 
Be also aware of the following: If you try to import a dataset which is older than the newest dataset imported, no values will be imported!
So make sure you import all datasets in the right order, going from oldest to newest.
\section{The config file (vcsd.cfg) }
\label{configfile}
% her er noe som kanskje skal inn ogs� 
%\subsection{XML\_to\_DB}
%The config file (vcsd.cfg) in /cfg is used to define the different filepaths and components which are going to be extracted to the database. The databaseinformation, namespace of XML-files, filepath of XML-files, and at least one component needs to be declared before using the XML to DB script. The one mandatory component in vcsd.cfg is an arbitrary child component from components/profile.
The config file can be edited with any editor.
Lines beginning with a # is ignored.
Consists of the following parts: 
\paragraph{DatabaseInfo}\\
Set the connection info to fit your mysql server.

\paragraph{PathToFiles}\\
This path should reflect the path to the dataset you want to import, 
for instance 
xmlpath=/home/user/xml-profiles/2008-03-05

\paragraph{Namespace}
Used by libXML to parse the data files. Default value is:
\\
namespace=http://www.lcfg.org/namespace/profile-1.0

\paragraph{Component}
This is the section where you specify which fields to import into the database.
The format is :\\ 
comp1 = inv/os\\ 

\paragraph{PreferredFields}
This section specifies which imported fields to use as a \"machine description\" in some visualisation techniques. Choose some values from the components section you configured. 

A sample configuration file cfg/vcsd.cfg is shown here:

\begin{verbatim}
#! VCSD Configuration file
# Configure with care
# This is just a sample file

<DatabaseInfo>
db=lcfg
dbtype=mySQL
dbhost=localhost
dbuser=username
dbpass=password
dbport=3306
</DatabaseInfo>

<PathToFiles>
#Uncomment one of these variables below

xmlpath=E:\mydocs\profiles\profiles-2008-03-05

</PathToFiles>

<Namespace>
namespace=http://www.lcfg.org/namespace/profile-1.0
</Namespace>

<Component>
#Which components to import from the XML files
#These components must be written like: comp<number>=comp/childcomp
# where <number> is an unique integer (doesn't need to be in order)
# and comp/childcomp is an XPath expression.  
#one profile component (such as profile/domain) is mandatory

comp1=inv/domain
comp2=inv/location
comp3=inv/manager
comp4=inv/model
comp5=inv/os
comp6=inv/owner
comp7=inv/sno
comp8=network/extrahosts
comp9=network/gateway
comp10=network/gatewaydev
comp17=xinetd/enableservices
comp22=profile/domain  #MANDATORY!
</Component>

<PreferredFields>
#The fields used to display information about one specific node.
# These fields will be collected out of the database generated, not the xml-files
# Hence it is important that these values also exist in the components section
prefield1=inv/manager
prefield2=inv/owner
prefield3=inv/location
prefield4=inv/sno
prefield5=inv/model
prefield6=inv/os
prefield7=network/gateway
#prefield8=profile/group
</PreferredFields>

\end{verbatim}


\section{Visualizing data}
\label{visualisation}
Here is something on how to visualize..

Prerequisites: A modern Web browser, octaga player. 
Fire up your browser and point it to http://localhost/cgi-bin/index.cgi

From the menu, choose a visualisation technique.
Depending on the technique chosen, choose the desired number of parameters and the desired fields to cluster on.
The visualisation will be embedded in the browser window, or you can open it manually from the output folder specified in vscd.cfg.

About navigation in the vrml browser... \\
This is how you do it.. \\ For further reference, please see www.octaga.no 


About the different techniques...\\

\chapter{System reference}

\section*{}
This part of the document will contain  documentation needed by developers for maintanence and expansion of the system

\newpage


\section{DataVisualizer}

\subsection{GUI}
The GUI consists of the following files and classes: 
\paragraph{cgiFunctions.pm}
Methods for printing HTML elements such as forms, stylesheets, menu, javascripts and embed visualisations.
Dependencies: DAL.pm

\paragraph{cgi scripts}\\
Index.cgi -- Start page \\
nodeVisualisation.cgi \\
pyramidVisualisation.cgi\\
groupVisualisation.cgi\\
spiralVisualisation.cgi\\
Dependencies: CGI

\subsection{BLL}
The business layer.
\subsubsection{Visualisation Library}
 Consists of a library of Visualisation modules, namely:
\subsubsection{GroupVisualizer}

\subsubsection{PyramidVisualizer}

\subsection{SpiralVisualizer}

\subsection{NodeVisualizer}

\subsection{PyramidVisualizer}
This module generates the vrml for the pyramid visualization. 
The pyramid visualizer module depends on the following modules: 

\subsubsection{Method descriptions}

\paragraph{new()}
This method is a constructor generating a new PyramidVisualizer object. Six parameters are required. 
The first three represent the table name, field name and field value for the first criteria query, while the last three represent the same values for the second criteria.

\paragraph{generateWorld()}
This method is responsible for generating the VRML-file for the visualization. 
\begin{itemize}
\item[-]Parameters: none
\item[-]Return value: String
\end{itemize}
First the data is retrieved from DAL, then the size of the steps are calculated before the string for the vrml file is bulit and returned.

\paragraph{hashValueAscendingNum}
Helping method, sorts a hash by its values in ascending order. Uses the global hash \verb"%steps"


\paragraph{hashValueDescendingNum}
Helping method, sorts a hash by its values in descending order. Uses the global hash \verb"%steps"


The visualizer modules depend on the following modules: 
\begin{itemize}
\item[-]
DAL.pm
\item[-]
vrmlGenerator.pm
\end{itemize}

\subsection{VRML\_Generator}

The VRML\_Generator is the largest module and is used by all the visualisation modules. Its main task is to generate valid VRML strings based on the attributes and method calls in the visualisation modules.
It is divided into different parts: 
\paragraph{Utility methods}\\
These are methods which can be used by any visualisation, such as setting color values, generating vector positions, printing routes and converting strings to "VRML-safe" syntax.

\paragraph{General VRML methods}
These methods are also generic methods which generates valid VRML code from desired parameters. 
A lot of common VRML nodes can be generated, including Timer, Transform, Group, Interpolator and Text. 
 
\paragraph{Proto methods}
Generates valid Proto nodes. (Static strings).
Proto nodes are definitions built by VRML nodes, fields and Scripts.
Used to define MachineNodes, viewpoints and menuitems and their attributes.

\paragraph{Specific methods for each Visualisation technique}
Methods used only by one specific visualisation module.

\subsection{DAL}
The Data Access Layer. \\
Files: DAL.pm \\
Connects to the database. \\
Dependencies: DBI\\

\section*{SRS System Requirement Specification}

\section*{Structure}
	Bilde / visualisering av komponenter og underprogrammer 
	\subsection*{Hardware requirements}
	
	Any newer computer (X86-compatible).

	\subsection*{Software requirements}

	
	Perl \verb#>=# 5.10 with XML::LibXML::XPathContext;
	For windows, the following packages are required to 		parse the XML files:
	\begin{itemize}
		\item[-]
		XML::LibXML
		\item[-]
		XML-LibXML-Common
		\item[-]
		XML-NamespaceSupport
		\item[-]
		XML::SAX 
	\end{itemize}	
	These packages can be retrieved from 
\verb"http://cpan.uwinnipeg.ca/PPMPackages/10xx/"	

	mySQL \verb#>=# 5.0.45
	apache webserver
	
	The client:
	A modern web browser -- tested in IE6, Mozilla Firefox, Opera, Safari .. bla
	VRML browser: recommended Octaga Player 
	
	
	

\section*{Database}

\section{Technologies}
This section will describe the technologies used throughout the project period to accomplish the final product.

\subsection{Perl}
Perl is dynamic programming language originally developed for fast text manipulation. It has been influenced by languages like C, AWK, and Lisp, and nowadays used for a wide range of tasks - e.g. network programming, GUI and web-programming, etc. Many operating systems are supported by Perl. The CPAN (Comprehensive Perl Archive Network) provides many third-party modules, which can improve and simplify various tasks.

\subsection{CGI}
The Common Gateway Interface (CGI) is a standard for interfacing external applications with an information server, e.g. HTTP servers. A CGI program is exectuted in real-time, giving the ability to display dynamic information.

Perl has CGI as a built-in module, and can use it to dynamically create web pages. 

\subsection{HTML}
The HyperText Markup Language (HTML) is used when publishing documents on the world wide web, and consists of codes which a browser interprets.

\subsection{VRML}
The Virtual Reality Modelling Language (VRML) is used to define three dimensional scenes, which allows a user to move around within its environment. VRML programs are event driven like HTML, and is commonly used to embed 3D effects on a web page. 

\subsection{MySQL}
The Structured Query Language (SQL) is a language that provides an interface to a database. MySQL is an open source software relational database management system, which is used to handle querys and transactions to a database.

Perl has a built-in module called DBI (database interface) which can interpret many types of structured query languages and

\subsection{Apache}
Apache is a web server which can be configured to run Perl::CGI and MySQL.


\section*{Installation}
Installation on Windows:

You need to install the following third-party software:

Apache, Perl, mySQL

We recommend installing activestate Perl -- link ,..

Install it to a folder, we will use \verb"C:\perl"

And then install WAMP server  -- -link

Unzip lcfgVisualizer.zip to a desired path, in our example we will use
\verb"C:/lcfgVisualiser/"

go to the folder and edit the .cgi files --- change the first line to reflect the location of your perl.exe file.

In this example, we will use \verb "#!C:\perl\bin\perl.exe -w"

To open for cgi execution in apache, you need the following lines in 
httpd.conf:

In the loadModule section, make sure these lines are not commented:
\begin{verbatim}
LoadModule alias_module modules/mod_alias.so
LoadModule cgi_module modules/mod_cgi.so
\end{verbatim}
in the section \verb"<IfModule alias_module>", add the following lines:
\begin{verbatim}
ScriptAlias /cgi-bin/ "C:/lcfgVisualizer/cgi/"

<Directory "C:/lcfgVisualizer/cgi/">
    AllowOverride None
    Options +ExecCGI Indexes
    Order allow,deny
    Allow from all
	AddHandler cgi-script .cgi .pl
</Directory>
\end{verbatim}

In the root directory of your apache server root ( default .... ), 
make a new folder named output
% vil ikke perl opprette katalogen for oss?
% i linux blir det kanskje et problem med rettigheter p� katalog og filer
in cgifunctions.pm
set 
\begin{verbatim}
\$FILEPATH = "E:\\www\\output\\";
\end{verbatim}


\section*{Bugs \& known issues}
BUG: Web browser freezes / crashes
GUI: On some systems, the octaga web browser plugin does not work.
Web browser: Internet Explorer 7.0.5730.11

Solution: Change browser -- use firefox.
Platform: 
Systems running Windows XP SP2, Internet Explorer 7.0.5730.11, Octaga player 2.2.0.12

Issue: Embedded vrml player spiser cpu, har 100 prosent i cpu.
Er dette pga skjermkortdriver feks?


Error hos meg: Embedded VRML worlds does not show up in firefox
Solved --- feil i stien til embeddinga, m� ta med .. foran  \$vrmlFile


BUG: Embedded VRML is not updated 
Firefox bruker cache selv om den ikke skal  det, funker ikke � force reload.
M� restarte browser for � f� vist en ny visualisering.
Forel�pig l�sning: ikke printe ut embeddinga, bare link,
dette hjelper ogs� mot at internet explorer henger hvis man pr�ver det i IE.


BUG: Some links inside the VRML world does not work
All links point to the same viewpoint. This is a anchor issue with name conventions.. 
- Viewer: Cortona 3D viewer
Platform: All
-Visualisations affected: groupVisualizer,...

Issue: touchSensor is misplaced in VRML Player on Mac OSX.
This means that if an object is linked with a touch sensor, one must click underneath the object to activate the sensor, rather than clicking the actual object. 

Issue: DataImporter. 
If you try to import a dataset of XML files which is older than the currently newest, nothing will be imported. 
Solution: Make sure you import datasets in a correct order, from oldest to latest.

BUG: Empty fields from the database in the CGI
When trying to visualize a field which has no values at all from the database, the user interface does not give any warnings about it. The select box in the web page is empty, and it is impossible to draw a VRML file.
Solution: Go back a few steps in the web page, and select another table.

