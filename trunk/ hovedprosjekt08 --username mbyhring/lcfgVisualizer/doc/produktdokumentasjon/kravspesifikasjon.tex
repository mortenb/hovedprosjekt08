\documentclass[11pt,a4paper]{article}
\usepackage[latin1]{inputenc}
\usepackage[norsk]{babel}
\usepackage{verbatim}
\usepackage{alltt}
\usepackage{graphicx}
%remove the following line if not using special symbols
\usepackage{amssymb}
\setlength{\parindent}{0pt}
\setlength{\parskip}{2ex plus 0.5ex minus 0.2ex}
\begin{document}
\title{Kravspesifikasjon}
\author{M Byhring, T E Iversen, L M Bredal\\
H�gskolen i Oslo, avdeling for ingeni�rutdanning}
\renewcommand{\today}{26 februar, 2008}
\maketitle

\newpage 

\section{Innledning}
\subsection{Om oppdragsgiver}
Oppdragsgivere er H�gskolen i Oslo ved lektor Simen Hagen, i samarbeid med
sluttbrukeren av systemet: Paul Anderson, professor ved University of Edinburgh. 

\subsection{Bakgrunn for prosjektet}
Paul Anderson har utviklet LCFG (Local ConFiGuration system), som er et system for � administrere konfigurasjon av et stort antall UNIX-maskiner. 
Hver unike maskin har en konfigurasjonsfil i XML-format, og Paul og Simen har tidligere diskutert muligheten for � lage en visuell oversikt over disse dataene. 

\section{Forord}
Denne kravspesifikasjonen beskriver de tekniske og funksjonelle kravene som skal oppfylles av hovedprosjektoppgaven "Visualizing System Configuration Data".
Kravene er utviklet av gruppemedlemmene selv, i samarbeid med oppdragsgiver og veileder, basert p� oppdragsgivers �nsker og interesser. 

\newpage
\setlength{\parskip}{0ex}
\tableofcontents
\setlength{\parskip}{2ex plus 0.5ex minus 0.2ex}
\newpage

\section{Kravspesifikasjon}
\subsection{Funksjonelle krav}
Systemet skal kunne:
\begin{itemize}
\item[$\triangleright$] Lese inn data fra XML-profilene og legge inn i en database.
\item[$\triangleright$] Hente spesifiserte dataefelt fra databasen.
\item[$\triangleright$] Visualisere data p� en eller flere hensiktsmessige m�ter.
\end{itemize}

\subsection{Tekniske krav}
\begin{itemize}

\item[$\triangleright$] Visualiseringen skal gj�res med VRML i kombinasjon med javascripts.
\item[$\triangleright$] Programmeringsspr�k skal v�re Perl til ekstrahering av data og generering av VRML.


\end{itemize}


\subsubsection{Krav til klient}
\label{klient}
\paragraph{Hardware}
\begin{itemize}
\item[$\triangleright$] En x86-kompatibel datamaskin
\item[$\triangleright$] CPU: Pentium 4 (2 GHz) eller bedre
\item[$\triangleright$] RAM: Minimum 1 GB
\item[$\triangleright$] OpenGL-kompatibelt skjermkort, minimum 128 MB dedikert minne

\end{itemize}

\paragraph{Software}
\begin{itemize}
\item[$\triangleright$] Octaga Player versjon $ >= $ 2.1
\item[$\triangleright$] Nettleser: Mozilla Firefox / Internet Explorer med st�tte for JavaScript.
\end{itemize}

\subsubsection{Krav til tjener}

\paragraph{Hardware}
\begin{itemize}
\item[$\triangleright$] En x86-kompatibel datamaskin
\item[$\triangleright$] CPU: Pentium-basert, 1 GHz eller bedre
\item[$\triangleright$] RAM: Minimum 512 MB
\item[$\triangleright$] Harddisk : Minimum 1 GB ledig plass.
\end{itemize}

\paragraph{Software}
\begin{itemize}
\item[$\triangleright$] Perl $>=$ 5.10 med n�dvendige biblioteker
\item[$\triangleright$] Apache webserver med CGI-st�tte
\item[$\triangleright$] mySQL $>=$ 5.0
\end{itemize}

\subsection{Kodestandard}
\begin{itemize}
\item[$\triangleright$] Alle Perl-moduler og scripts som utvikles skal bruke streng syntaks. (use strict).
\item[$\triangleright$] Kommentarer i kildekoden skal v�re p� engelsk.
\item[$\triangleright$] Den genererte VRML-koden skal f�lge gyldig VRML 2.0 standard. 
\item[$\triangleright$]Generert kildekode skal v�re fullstendig fri for b�de syntaktiske og semantiske feil, og interaksjon og animasjon skal fungere korrekt p� et klient-system som oppfyller systemkravene gitt i \ref{klient}.
\end{itemize}

\subsection{Krav til dokumentasjon}
\begin{itemize}
\item[$\triangleright$] Brukerdokumentasjon, kildekode samt deler av sluttdokumentasjonen m� v�re tilgjengelig p� engelsk.
\end{itemize}

\subsection{Andre krav}
\begin{itemize}
\item[$\triangleright$] Systemet skal v�re tilgjengelig som �pen kildekode.
\item[$\triangleright$] Systemet skal v�re plattformuavhengig. 
\item[$\triangleright$] Systemet skal v�re generisk og utvidbart.

\end{itemize}


\end{document}