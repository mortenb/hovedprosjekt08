\documentclass[12pt,a4paper]{report}
\title{Product Documentation}

\begin{document}
\section*{Foreword}
hei
\section*{Description of the program}

LCFGvisualizer is a program..

The program is divided into two seperate parts:
dataImporter and dataVisualizer (figure ..) image

1. dataImporter
This is the part used to parse the XML-profiles, extract the wanted information and export it to a mySQL database. 

2. dataVisualizer
This part connects to the database and generates a visualisation based on which criterias and techiques the user sets.


\\
Here be more dragons

\section*{SRS System Requirement Specification}
\section*{Structure}
	Bilde / visualisering av komponenter og underprogrammer 
	\subsection*{Hardware requirements}
	
	Any newer computer (X86-compatible).

	\subsection*{Software requirements}

	
	Perl >= 5.10 with XML::LibXML::XPathContext;
		
	mySQL >= 5.0.45
	apache webserver
	
	The client:
	A modern web browser -- tested in IE6, Mozilla Firefox, Opera, Safari .. bla
	VRML browser: recommended Octaga Player 
	
	
	

\section*{Database}


\section*{Installation}
Installation on Windows:

You need to install the following third-party software:

Apache, Perl, mySQL

We recommend installing activestate Perl -- link ,..

Install it to a folder, we will use C:\perl

And then install WAMP server  -- -link

Unzip lcfgVisualizer.zip to a desired path, in our example we will use
C:/lcfgVisualiser/

go to the folder and edit the .cgi files --- change the first line to reflect the location of your perl.exe file.

In this example, we will use #!C:\perl\bin\perl.exe -w

To open for cgi execution in apache, you need the following lines in 
httpd.conf:

In the loadModule section, make sure these lines are not commented:

LoadModule alias_module modules/mod_alias.so
LoadModule cgi_module modules/mod_cgi.so

in the section <IfModule alias_module>
ScriptAlias /cgi-bin/ "C:/lcfgVisualizer/cgi/"


<Directory "C:/lcfgVisualizer/cgi/">
    AllowOverride None
    Options +ExecCGI Indexes
    Order allow,deny
    Allow from all
	AddHandler cgi-script .cgi .pl
</Directory>

In the root directory of your apache server root ( default .... ), 
make a new folder named output

in cgifunctions.pm
set 
$FILEPATH = "E:\\www\\output\\";

\end{document}