\documentclass[12pt,a4paper,onecolumn]{article}
\usepackage[norsk]{babel}
\usepackage[latin1]{inputenc}
%\usepackage[T1]{fontenc}
\usepackage{babel, textcomp}
\usepackage{verbatim}
\usepackage{graphicx}
\usepackage{url}
\usepackage{enumerate}
\usepackage{verbatim}
\usepackage{alltt}
\usepackage{graphicx}

%\usepackage{setspace}
%\doublespacing
%\onehalfspacing

\tolerance = 5000      % LaTeX er normalt streng n�r det gjelder
                      % linjebrytingen.
\hbadness = \tolerance % Vi vil v�re litt mildere, s�rlig fordi norsk
                      % har s�
\pretolerance = 2000   % mange lange sammensatte ord.

\begin{document}


%remove the following line if not using special symbols

\title{Kravspesifikasjon}
\author{Lars Martin Bredal \and Morten Byhring \and Tom Erik Iversen \and
H�gskolen i Oslo, avdeling for ingeni�rutdanning}
\maketitle

\newpage 
\section{Forord}
Denne kravspesifikasjonen beskriver de tekniske og funksjonelle kravene som skal oppfylles av hovedprosjektoppgaven "Visualizing System Configuration Data".
Kravene er utviklet av gruppemedlemmene selv, i samarbeid med oppdragsgiver og veileder, basert p� oppdragsgivers �nsker og interesser. 

\newpage

\tableofcontents

\newpage 

\section{Innledning}
\subsection{Om oppdragsgiver}
Oppdragsgivere er H�gskolen i Oslo ved lektor Simen Hagen, i samarbeid med
sluttbrukeren av systemet: Paul Anderson, professor ved University of Edinburgh. 

\subsection{Bakgrunn for prosjektet}
Paul Anderson har utviklet LCFG (Local ConFiGuration system), som er et system for � administrere konfigurasjon av et stort antall UNIX-maskiner. 
Hver unike maskin har en konfigurasjonsfil i XML-format, og Paul og Simen har tidligere diskutert muligheten for � lage en visuell oversikt over disse dataene. 


\newpage

\section{Kravspesifikasjon}
\subsection{Funksjonelle krav}
Systemet skal kunne:
\begin{itemize}
\item Lese inn data fra XML-profilene og legge disse inn i en database
\item Hente spesifiserte datafelt fra databasen
\item Visualisere data p� en eller flere hensiktsmessige m�ter
\end{itemize}

\subsection{Tekniske krav}
\begin{itemize}

\item Visualiseringen skal gj�res med VRML i kombinasjon med javascripts
\item Programmeringsspr�k skal v�re Perl til ekstrahering av data og generering av VRML


\end{itemize}


\subsubsection{Krav til klient}
\label{klient}
%\begin{Hardware} %-> \item[Hardware]
\paragraph{Hardware}
\begin{itemize}
\item CPU: Pentium 4 (1.5 GHz) eller bedre
\item RAM: Minimum 1 GB
\item OpenGL-kompatibelt skjermkort, minimum 32 MB dedikert minne

\end{itemize}

\paragraph{Software}
\begin{itemize}
\item Octaga Player versjon $ >= $ 2.1
\item Nettleser: Mozilla Firefox / Internet Explorer med st�tte for JavaScript
\end{itemize}

\subsubsection{Krav til tjener}
\paragraph{Hardware}
\begin{itemize}
%\begin{itemize}

\item CPU: Pentium-basert, 1 GHz eller bedre
\item RAM: Minimum 512 MB
\item Harddisk : Minimum 1 GB ledig plass
%\end{itemize}
\end{itemize}

\paragraph{Software}
\begin{itemize}
\item Perl $>=$ 5.10 med n�dvendige biblioteker
\item Apache webserver med CGI-st�tte
\item MySQL $>=$ 5.0
\end{itemize}

\subsection{Kodestandard}
\begin{itemize}
\item Alle Perl-moduler og scripts som utvikles skal bruke streng syntaks
\verb#(use strict)#
\item Kommentarer i kildekoden skal v�re p� engelsk
\item Den genererte VRML-koden skal f�lge gyldig VRML 2.0-standard 
\item Generert kildekode skal v�re fullstendig fri for b�de syntaktiske og semantiske feil, og interaksjon og animasjon skal fungere korrekt p� et klientsystem som oppfyller systemkravene gitt i \ref{klient}
\end{itemize}

\subsection{Krav til dokumentasjon}
\begin{itemize}
\item Brukerdokumentasjon, kildekode samt deler av sluttdokumentasjonen m� v�re tilgjengelig p� engelsk
\end{itemize}

\subsection{Andre krav}
\begin{itemize}
\item Systemet skal v�re tilgjengelig som �pen kildekode
\item Systemet skal v�re plattformuavhengig
\item Systemet skal v�re generisk og utvidbart

\end{itemize}


\end{document}